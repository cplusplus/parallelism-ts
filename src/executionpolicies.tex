%!TEX root = ts.tex

\rSec0[parallel.execpol]{Execution policies}

\rSec1[parallel.execpol.synopsis]{Header \tcode{<experimental/execution>} synopsis}

\begin{codeblock}
#include <execution>

namespace std::experimental {
inline namespace parallelism_v2 {
namespace execution {
  // \ref{parallel.execpol.unseq}, Unsequenced execution policy
  class unsequenced_policy;

  // \ref{parallel.execpol.vec}, Vector execution policy
  class vector_policy;

  // \ref{parallel.execpol.objects}, Execution policy objects
  inline constexpr unsequenced_policy unseq{ @\unspec@ };
  inline constexpr vector_policy vec{ @\unspec@ };
}
}
}
\end{codeblock}

\rSec1[parallel.execpol.unseq]{Unsequenced execution policy}

\begin{codeblock}
class unsequenced_policy { @\unspec@ };
\end{codeblock}

\pnum
The class \tcode{unsequenced_policy} is an execution policy type used as
a unique type to disambiguate parallel algorithm overloading and indicate that
a parallel algorithm's execution may be vectorized, e.g., executed on a single
thread using instructions that operate on multiple data items.

\pnum
The invocations of element access functions in parallel algorithms
invoked with an execution policy of type \tcode{unsequenced_policy} are
permitted to execute in an unordered fashion in the calling thread, unsequenced
with respect to one another within the calling thread. \begin{note}This means
that multiple function object invocations may be interleaved on a single
thread.\end{note}

\pnum
\begin{note}
  This overrides the usual guarantee from the \Cpp Standard, \CppXref{intro.execution} that function executions do not overlap with one another.
\end{note}

\pnum
During the execution of a parallel algorithm with the \tcode{experimental::execution::unsequenced_policy} policy, if the invocation of an element access function exits via an uncaught exception, \tcode{terminate()} will be called.

\rSec1[parallel.execpol.vec]{Vector execution policy}

\begin{codeblock}
class vector_policy { @\unspec@ };
\end{codeblock}

\pnum
The class \tcode{vector_policy} is an execution policy type used as a
unique type to disambiguate parallel algorithm overloading and indicate that a
parallel algorithm's execution may be vectorized. Additionally, such
vectorization will result in an execution that respects the sequencing
constraints of wavefront application
(\hyperref[parallel.alg.general.wavefront]{[parallel.alg.general.wavefront]}).
\begin{note}The implementation thus makes stronger guarantees than for
\tcode{unsequenced_policy}, for example.\end{note}

\pnum
The invocations of element access functions in parallel algorithms
invoked with an execution policy of type \tcode{vector_policy} are permitted to
execute in unordered fashion in the calling thread, unsequenced with respect to
one another within the calling thread, subject to the sequencing constraints of
wavefront application
(\hyperref[parallel.alg.general.wavefront]{[parallel.alg.general.wavefront]})
for the last argument to \tcode{for_loop}, \tcode{for_loop_n},
\tcode{for_loop_strided}, or \tcode{for_loop_strided_n}.

\pnum
During the execution of a parallel algorithm with the
\tcode{experimental::execution::vector_policy} policy, if the invocation of an
element access function exits via an uncaught exception, \tcode{terminate()}
will be called.

\rSec1[parallel.execpol.objects]{Execution policy objects}

\begin{codeblock}
inline constexpr execution::unsequenced_policy unseq{ @\unspec@ };
inline constexpr execution::vector_policy vec{ @\unspec@ };
\end{codeblock}

\pnum
The header \tcode{<experimental/execution>} declares a global object
associated with each type of execution policy defined by this Technical
Specification.

